\documentclass[]{base/exam}

\examtype{Abschlussklasur}
\course{IT-Sicherheit}
\semster{Wintersemester 2017/18}
\examdatetime{06.02.2018}

\generalhints{
    \begin{itemize}
        \item Sie haben \textbf{90 Minuten} Zeit.
        \item Die Klausur besteht aus \textbf{\totalExercises{} Aufgaben }und insgesamt \textbf{\totalExercisePoints{} Punkten}.
        \item Es sind \textbf{keinerlei Hilfsmittel} erlaubt.
        \item Schalten Sie \textbf{alle Elektronischen Geräte aus}, insbesondere auch Smartphones, Smart Watches, Taschenrechner und alle Kommunikationsgeräte.
        \item Benutzen Sie einen Kugelschreiber. \textbf{Abgaben, die mit Bleistift geschrieben sind, werden nicht gewertet}.
        \item Füllen Sie die \textbf{obere Tabelle} aus.
        \item Schreiben Sie Ihre Matrikelnummer auf \textbf{oben jedes Blatt}
        \item Wenn Ihnen der Platz nicht reichen sollte, können Sie Papier von einem Tutor erhalten. \textbf{Verwenden Sie keine eigenes Papier!}
        \item Halten Sie ein \textbf{Ausweisdokument} und Ihren \textbf{Studierendenausweis} bereit.
        \item Wenn Sie nicht möchten, dass Ihre Matrikelnummer in der Ergebnisliste auftaucht, \textbf{geben Sie bitte ein Codewort an}, das stattdessen genutzt werden soll, ansonsten lassen Sie das Feld bitte leer.
    \end{itemize}
}
    
\begin{document}
    \Exercise{Safety vs. Security}
\Task{1}{Erklären Sie kurz und prägnant den Unterschied zwischen Safety und Security.}
\begin{solution}
	Ausfallsicherheit und Einbruchsicherung
	\begin{itemize}
		\item Safety: Betriebs- / Ausfallsicherheit, Sicherung gegenüber Unfällen.
		\item Security: Kriminalprävention, Sicherung gegenüber Angriffen.
	\end{itemize}
\end{solution}

\Task{1}{Nennen Sie jeweils ein Beispiel einer entsprechenden Bedrohung.}
\begin{solution}
	Notausgangstür
	\begin{itemize}
		\item Safety: Die Tür muss zu jedem Zeitpunkt die \textbf{Flucht} aus dem Gebäude \textbf{ermöglichen}.
		\item Security: Die Tür muss \textbf{Unbefugten} den \textbf{Zutritt} zu jedem Zeitpunkt \textbf{verwären}.
	\end{itemize}
\end{solution}

\Task{1}{Was ist häufig das Problem mit diesen Begriffen in unserer Sprache / im Alltag?}
\begin{solution}
	Personen können beim Verwenden des Wortes verschiedene Verständnisse des Begriffs haben. Gerade in der Domäne Informatik ist Aussfallsicherheit und Sicherheit gegenüber Angriffen häufig gemeinsames Ziel, z.B. bei Servern. Soll die Sicherheit erhöht werden, können Auftraggeber und -nehmer also sehr leicht aneinander vorbei reden. 
\end{solution}
~
\end{document}    