\documentclass[]{base/exercise}

\course{IT-Sicherheit}
\semster{Wintersemester 2017/18}
\returndate{\emph{keine Abgabe}}
\exercisenumber{0}

\begin{document}

\Exercise{Anmeldung im Ilias}
Melden Sie sich im Ilias an, damit Sie Übungsblätter, Folien, etc. zur Vorlesung herunterladen können. Bitte geben Sie dabei neben einer gültigen Email Adresse (im Idealfall die, mit der Sie auch Abgaben per Mail schicken würden) auch Ihren Studiengang und ihr aktuelles Semester an (aus statistischen Gründen).

~

\Exercise{Erstellung einer Linux VM}
\Task{0}{Installation der VM}
Den praktischen Teil des Python Crashkurses und die meisten Programmieraufgaben auf den Zetteln lassen sich am besten in einer Linux VM lösen. Richten Sie sich daher eine virtuelle Maschine mit Linux ein. Ob Sie dafür den \emph{VMware Player}, \emph{VirtualBox} oder einen anderen Hypervisor verwenden bleibt Ihnen überlassen.
\begin{itemize}
	\item Die Auswahl der Linux-Distribution ist Ihnen überlassen, allerdings empfehlen wir Ubuntu oder Debian, weil wir bei Fragen dann am besten helfen können.
	\item Wir empfehlen weiterhin die Ubuntu Long-Term-Support (LTS) Version (16.04), weil die aktuelle Version (17.04) nur noch bis Januar mit Updates versorgt wird.
	\item Für andere Systeme können wir bei Problemen im schlimmsten Fall nicht weiterhelfen.
\end{itemize}

\Task{0}{Python}
Stellen Sie sicher, dass in der VM eine Python-Umgebung installiert ist. Python 2.x ist auf Ubuntu standardmäßig vorhanden. Installieren Sie in Ihrer VM zusätzlich das Paket build-essential, das den C-Compiler, make, gdb und weitere Entwicklungswerkzeuge enthält.

\Task{0}{Editor}
Installieren Sie einen Editor, mit dem Sie gut arbeiten können. Wer keine Erfahrung mit \emph{vi} oder \emph{emacs} hat, kann z.B. \emph{Scite}, einen einfachen grafischen Editor, oder \emph{Atom} (\url{http://atom.io}), einen ziemlich gut erweiterbaren Editor mit Fokus auf Entwicklung. Eine weitere (kostenpflichtige) Möglichkeit wäre \emph{Sublime Text}.

~
\Exercise{Machen Sie sich mit Linux vertraut}
Spielen Sie ein bisschen mit der VM herum, insbesondere mit dem Terminal und der Bash. Testen Sie ggf. einige Teile des Python Codes auf den Folien des Crashkurses in Ihrer VM.

\end{document}
